\documentclass{beamer}
%\usepackage[T1]{fontenc}
%\usepackage[utf8]{inputenc}
\usepackage{qtree}
\usepackage{amsmath}
\usepackage{verbatim}
\usepackage{skak}
\usepackage{graphicx}
\usepackage{times}

\usetheme{default}
\setbeamercovered{highly dynamic}

\newcounter{saveenumi}
\newcommand{\seti}{\setcounter{saveenumi}{\value{enumi}}}
\newcommand{\conti}{\setcounter{enumi}{\value{saveenumi}}}


\title[Tips and Tricks]
{The need to automatize tasks}
\subtitle{}
\author[Christophe Pallier] % (optional, for multiple authors)
{Christophe Pallier}
\institute
{
  CNRS\\
  Unit\'e INSERM-CEA de Neuroimagerie Cognitive\\
  Gif-sur-Yvette
}
\date[17 Juin 2013] % (optional)

\begin{document}

\frame{\titlepage}

\begin{frame}
\frametitle{Table of Contents}
\tableofcontents[currentsection]


\end{frame}

\resetcounteronoverlays{saveenumi}

% \begin{frame}[<+->]
%   \begin{enumerate}
%   \item foo
%   \item bar%
%     \seti
%   \end{enumerate}
% \end{frame}

% \begin{frame}[<+->]
%   \begin{enumerate}
%     \conti
%   \item zip
%   \item yadda%
%     \seti
%   \end{enumerate}
% \end{frame}

\section{Why?}

\begin{frame}
\frametitle{Why a workshop on Computer Science tools for Cognitive Science?}

\begin{itemize}
\item Perform simulations
\pause

\item Selecte stimuli in databases, or generating them
\pause

\item Create experimental lists (distribution of conditions, order of trials...)
\pause

\item Stimulate participants and record their responses
\pause

\item Analyse Data (Reaction times, EEG, fMRI)
\pause

\item Generate Reports/publication quality figures
\pause

\item ??? 
\end{itemize}

\pause

\begin{centering}
``Either you are the slave of the computer or the computer is your slave''
\end{centering}
\end{frame}

\section{Reproducible Science}

\begin{frame}
\frametitle{Reproducible Science}

You should strive to make your experiments and analyses reproducible... 
by others, but also by yourself!

\pause

{ \fontsize{9pt}{11}\selectfont

\begin{itemize}
\item you should keep track of exactly how you selected your materials
\item you should keep track of what you did exactly for the analyses
\item someone else should be able to check what you did, and reproduce it
\item This is often very difficult to achieve!
\end{itemize} }

\pause
Possible strategies:

{ \fontsize{9pt}{11}\selectfont


\begin{enumerate}
\item keep a detailed lab notebook (I only know one person who can do it) 
\item write computer programs that can entirely reproduce your experiments and your analyses
\item give up,  hope you have not made mistakes, and will not need to check or rerun the experiment
\end{enumerate}
}

\end{frame}

\begin{frame}
\frametitle{Tools for reproducible science}


\begin{itemize}
  \item It is worth learning how to program cleanly! The aim is not simply to write a program that works but a program that can be reread and modified. In the end, you will spend less time in front of the computer

\pause

  \item Programming tools

    \begin{itemize}
    \item Good ones: Python, R, Matlab ...

    \item Less good ones: Excel, E-prime...

      \begin{itemize}
      \item impossible to check thouroughly.
      \item compatibility not assured between successive versions.
      \item they have their use notheless.
        
      \end{itemize}
    \end{itemize}

\pause

  \item Version control tools (git, mercurial, svn...) allow to 
    keep track of the history of all files and (b) facilitate collaboration between several people
  
\pause
\item Check lessons on http://software-carpentry.org/v4/index.html

\item {\small Open an account on github.com; create a new repository; install a git client on your computer; clone the repository; work on it, add and commit files, and pull them back to the github repo.}

\end{itemize}



\end{frame}

%%%%%%%%%%%%%%%%%%%%%%%%%%%%%%%%%%%%%%%%%%%%%%%%%%%%%%%%%%%%%%%%%%%%%%%%

\begin{frame}[fragile]
\frametitle{An example: Selecting nouns and verbs for an experiment}

Suppose you need to select nouns and verbs that are 4 phonemes long and have 4-6 letters.

\begin{enumerate}
\item You can go to www.lexique.org and user the interface to obtain such lists.

\item (better) Download the current database and write a script to select your materials.
\end{enumerate}

See demo in lexique\_search

\end{frame}




\end{document}
