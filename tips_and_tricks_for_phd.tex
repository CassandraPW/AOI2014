\documentclass{beamer}
%\usepackage[T1]{fontenc}
%\usepackage[utf8]{inputenc}
\usepackage{qtree}
\usepackage{amsmath}
\usepackage{verbatim}
\usepackage{skak}
\usepackage{graphicx}
\usepackage{times}

\usetheme{default}
\setbeamercovered{highly dynamic}

\newcounter{saveenumi}
\newcommand{\seti}{\setcounter{saveenumi}{\value{enumi}}}
\newcommand{\conti}{\setcounter{enumi}{\value{saveenumi}}}


\title[Tips and Tricks]
{Tips and Tricks for Ph.D. students}
\subtitle{}
\author[Christophe Pallier] % (optional, for multiple authors)
{Christophe Pallier}
\institute
{
  CNRS\\
  Unit\'e INSERM-CEA de Neuroimagerie Cognitive\\
  Gif-sur-Yvette
}
\date[17 Juin 2013] % (optional)

\begin{document}

\frame{\titlepage}

\begin{frame}
\frametitle{Table of Contents}
\tableofcontents[currentsection]


\end{frame}

\resetcounteronoverlays{saveenumi}

% \begin{frame}[<+->]
%   \begin{enumerate}
%   \item foo
%   \item bar%
%     \seti
%   \end{enumerate}
% \end{frame}

% \begin{frame}[<+->]
%   \begin{enumerate}
%     \conti
%   \item zip
%   \item yadda%
%     \seti
%   \end{enumerate}
% \end{frame}



\section{Reproducible Science}

\begin{frame}
\frametitle{Reproducible Science}

You should strive to make your experiments and analyses reproducible... 
by others, but also by yourself!

\pause

{ \fontsize{9pt}{11}\selectfont

\begin{itemize}
\item you should keep track of exactly how you selected your materials
\item you should keep track of what you did exactly for the analyses
\item someone else should be able to check what you did, and reproduce it
\item This is often very difficult to achieve!
\end{itemize} }

\pause
Possible strategies:

{ \fontsize{9pt}{11}\selectfont


\begin{enumerate}
\item keep a detailed lab notebook (I only know one person who can do it) 
\item write computer programs all the processing pipelines
\item give up,  hope you have not made mistakes, and will not need to check or rerun the experiment
\end{enumerate}
}

\end{frame}

\begin{frame}
\frametitle{Tools for reproducible science}


\begin{itemize}
  \item It is worth learning how to program cleanly! . The aim is notsimply to write a program that works but a program that can be reread and modified. In the end, you will spend less time in front of the computer

\pause

  \item Programming tools

    \begin{itemize}
    \item Good ones: Python, R, Matlab ...

    \item Bad ones: Excel, E-prime...

      \begin{itemize}
      \item impossible to check thouroughly.
      \item compatibility not assured between successive versions.
      \item it is not impossible to make good use of Excel and Eprime
        
      \end{itemize}
    \end{itemize}

\pause

  \item Version control tools (svn, git, mercurial,...)

    \begin{itemize}
    \item keep track of the history of a files (all previous versions)
    \item allow to collaborate between several people
    \end{itemize}

  \item Suggested site ``Software Carpentry''

\end{itemize}



\end{frame}

%%%%%%%%%%%%%%%%%%%%%%%%%%%%%%%%%%%%%%%%%%%%%%%%%%%%%%%%%%%%%%%%%%%%%%%%

\begin{frame}[fragile]
\frametitle{Selecting materials from Lexique for an experiment}

You should not use Lexique's web interface but download the current database and write a script to select your materials.

See demo in lexique\_search

\end{frame}

%%%%%%%%%%%%%%%%%%%%%%%%%%%%%%%%%%%%%%%%%%%%%%%%%%%%%%%%%%%%%%%%%%%%%%%%

\begin{frame}
\frametitle{Data analysis with R}

\centering
\begin{tabular}{cc}
\textbf{\textsf{Left Hemisphere}} & \textbf{\textsf{Right Hemisphere}} \\
\multicolumn{2}{l}{\textbf{\textsf{Musicians}}} \\

\quad \includegraphics[width=0.4\textwidth]{musicians_L_s2.pdf}  &
\includegraphics[width=0.4\textwidth]{musicians_R_s2.pdf} \\
	
\multicolumn{2}{l}{\textbf{\textsf{Non-Musicians}}} \\

\quad \includegraphics[width=0.4\textwidth]{non_musicians_L_s2.pdf}  &
\includegraphics[width=0.4\textwidth]{non_musicians_R_legend_s2.pdf} \\
									
\end{tabular}

{\footnotesize Amplitudes of the constituent size effect}

\end{frame}


\begin{frame}
\frametitle{Beyond p-values!!!}

``Certain journals present tables of p-values (or even worse, F-statistics, degrees of freedom and associated p-values)'' Gerald van Belle \emph{Statistical rules of thumb} Wiley.

Rule of thumb: Show your data! Report your results with estimates of effects and the associated confidence intervals.


\vspace*{2cm}

See also G. Loftus (1996) Psychology will be a Much Better Science
When We Change the Way We Analyze Data. \emph{Current directions in Psychological Science}.

\end{frame}

\begin{frame}
\frametitle{One of the problem with significance tests}

{\footnotesize (from Gelmann \& Stern (2005). The difference between ``significant'' and ``not significant'' is not itself statistically significant.)}

\vspace*{12pt}

Consider two independent studies estimating the same effect:

\quad  $\delta_1 = 25 \pm 10$ ($p<.01$)

\quad  $\delta_2 = 10 \pm 10$ ($p>.10$)

\vspace*{6pt}

  It would be tempting to conclude that there is a large difference
  between the two studies. However, the difference ($15 \pm 14$) is not
  even close to being statistically significant.

\vspace*{6pt}

\pause
  Now imagine a third replication:

\quad  $\delta_3 = 2.5 \pm 1.0$ ($p<.01$)

  This third study attains the same significance level as the first
  study, yet the difference between the two is itself also
  significant!




\end{frame}




\section{Writing}


\begin{frame}[<+->]
\frametitle{Writing}

\begin{itemize}

\item Writing is a matter of successive refinement:

To write something, you must first write something dirty and the clean it.
You should write a first draft, a second draft,... to obtain the final version.

\item For most people, writing is very difficult. For me, I find it easier if I can work *continuously*. I have a huge cost of starting again.

\item For the PhD, I recommend to start writing 1 year before the deadline.

\item I find PhD manuscripts based on papers frustrating (even if I recognize it is efficient)

\item peril of perfectionism

\item learn touch-typing (two persons I know who did it: Stan Dehaene \& Anne Christophe) 

\end{itemize}  

\end{frame}


\begin{frame}[<+->]
\frametitle{Tools for writing}

\begin{quote}
``Linguistics is cheap: you just need a pen and an eraser. Philosophy is even cheaper: you do not need the eraser''
\end{quote}

We, psychologists, need more technology...

Word or \LaTeX{}? That is the question...

My reasons to use \LaTeX{}:

\begin{itemize}
\item  very bad experiences with Word/OpenOffice (crashes, bugs). Never lost work with LaTeX

\item  produces tidy complex documents with typically less work than Word (if one refrains from customizing)

\item  allows one to automatically generate documents (particularily useful for graphics)

\item drawback: like programming: you have to learn a language.
\end{itemize}
\end{frame}


\begin{frame}[fragile]
\frametitle{LaTeX example 1: syntactic trees}
\fontsize{6pt}{7.2}\selectfont

\Tree [.IP [.Adjunct 
              Even
              [.CP 
               [.C if ]  
               [.IP 
                 [.DP [.D the ] [.NP$_1$ kids ]]
                 [.VP [.V spoke ] [.Adv loudly ]]
               ]]] 
            [.IP 
                [.DP [.D their ] [.NP$_2$ parents ]] 
                [.VP slept ]]]

\vspace*{12pt}
\hrule

\begin{verbatim}
\Tree [.IP [.Adjunct  Even [.CP 
               [.C if ]  
               [.IP [.DP [.D the ] [.NP_1 kids ]]
                    [.VP [.V spoke ] [.Adv loudly ]]]]] 
            [.IP 
                [.DP [.D their ] [.NP_2 parents ]] 
                [.VP slept ]]]
\end{verbatim}

\end{frame}


\begin{frame}[fragile]
\frametitle{LaTeX example 2}

\newenvironment{bcases}
  {\left\lbrace\begin{aligned}}
  {\end{aligned}\right\rbrace}

\centering 

Even if the 
$\begin{bcases}
\mbox{kids}\\
\mbox{naughty kids}\\
\mbox{very naughty kids}\\
\end{bcases}$
spoke loudly, ...

\vspace*{1cm}
\hrule

\fontsize{8pt}{10}\selectfont


\begin{verbatim}

Even if the $\begin{bcases}
   \mbox{kids}              \\
   \mbox{naughty kids}      \\
   \mbox{very naughty kids} \\
\end{bcases}$ spoke loudly, ...

\newenvironment{bcases}
  {\left\lbrace\begin{aligned}}
  {\end{aligned}\right\rbrace}

\end{verbatim}

\end{frame}

\begin{frame}[fragile]
\frametitle{LaTeX example 3}

\begin{center}
\includegraphics[width=0.5\textwidth]{chess.png}
\end{center}

\vspace*{1cm}

\hrule

\fontsize{6pt}{7}\selectfont

The code uses the well-established Forsyth-Edwards Notation:

\begin{verbatim}
\fenboard{r5k1/1b1p1ppp/p7/1p1Q4/2p1r3/PP4Pq/BBP2b1P/R4R1K w - - 0 20}
\end{verbatim}

\end{frame}

\begin{frame}[fragile]
\end{frame}

%%%%%%%%%%%%%%%%%%%%%%%%%%%%%%%%%%%%%%%%%%%%%%%%%%%%%%%%%%%%%%%%%%%%%%%%

\section{The ideal PhD}

\begin{frame}[<+->]
\frametitle{My point of view about the ideal  PhD}

\addtolength{\itemsep}{2\baselineskip}

\begin{enumerate}
\item A PhD Candidate is not a Research Assistant.

  {\small 
    It should not even be called a ``student''. 

    It is a young researcher who still has to acquire some technical
    and scientific knowledge, but who should already have the mindset
    of a researcher (curiosity and rational thinking)}

\item ``Directeur de th\`ese'' vs. ``PhD advisor''.

I see my role as:

  {\small
  \begin{itemize}
  \item provides the PhD with the means to perform the research
  \item provide intellectual guidance and councelling
  \item show how to do things.
  \end{itemize}
  }
\seti
\end{enumerate}
\end{frame}

\begin{frame}
\frametitle{My point of view about the ideal  PhD}

\begin{enumerate}
\conti
\item  What I expect from the PhD ``student'':

  {\small
  \begin{itemize}
  \item the student should progressively become the ``master'' of the project.
  \item s/he should think by herself/himself. We should have two-way
    exchanges and become colleagues.
  \item read the litterature. know what s/he knows and what s/he does
    not know.
  \item report  when one is blocked.
  \end{itemize}
  }

\pause

\item A PhD can mean:  

  {\small 
  \begin{itemize}
  \item 3 years of quasi total freedom to investigate a question that
    interest you (this was the case in Jacques Mehler's lab) 
  \item 3 years of painful work if you do not understand/like the
    topic, are obsessed with getting results, etc...
  \end{itemize}
  }

\end{enumerate}
 
\end{frame}

\begin{frame}{If you need to improve your work organization}
\frametitle{}

If you feel overwhelmed and inefficient; if the stress in front of the
many takss is paralyzing you, offer yourself some useful
procrastination:

\begin{itemize}

\item Getting Things Done (GTD): Getting Things Done: The Art of
    Stress-Free Productivity

\item Zen to done (ZTD): The Ultimate Simple Productivity
    System

\item Learn about \textbf{Mind Mapping} (Note-taking that maps out your ideas)
\end{itemize}

\end{frame}

%%%%%%%%%%%%%%%%%%%%%%%%%%%%%%%%%%%%%%%%%%%%%%%%%%%%%%%%%%%%%%%%%%%%%%%%


%%%%%%%%%%%%%%%%%%%%%%%%%%%%%%%%%%%%%%%%%%%%%%%%%%%%%%%%%%%%%%%%%%%%%%%%

\begin{frame}
\frametitle{On the importance of being a bilingual in Science}

\begin{center}
\includegraphics[width=0.66\textwidth]{shark.png}
\end{center}

\vspace*{2cm}

Bela Julesz (a hungarian psychologist-engineer) claimed that
``scientific bilingualism'' is the key to creative contributions to
science.

For example, random dot stereograms were invented by an engineer who
knew that camouflage does not exist in 3D, contradicting psychological
theories of stereopsis.

\end{frame}


%%%%%%%%%%%%%%%%%%%%%%%%%%%%%%%%%%%%%%%%%%%%%%%%%%%%%%%%%%%%%%%%%%%%%%%%

\begin{frame}
\frametitle{A bit of Epistemology can do no harm}

\begin{itemize}

\item Sometimes, When I read (neuro)cognitive papers, I miss behaviorism.

\item Suggested Reading:\\ \quad Zolt\'an Dienes \emph{Understanding Psychology as a Science}

\item This diagram may be banal, but worth showing anyway:

\begin{center}
\includegraphics[width=0.5\textwidth]{Diagram1.pdf}
\end{center}


Note: It is crucial that the prediction be made before the observation! (many papers are full of post-hoc explanations)

\fontsize{8pt}{10}\selectfont

\item Remark: To locate a failure in a broken equipment or debug a program, it is exactly the same approach. Yet, many people can't seem to follow the approach. See Tatham's \emph{How to Report Bugs Effectively}.

\end{itemize}

\end{frame}



\end{document}